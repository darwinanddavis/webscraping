\documentclass[10,portrait]{article}
\usepackage{lmodern}
\usepackage{amssymb,amsmath}
\usepackage{ifxetex,ifluatex}
\usepackage{fixltx2e} % provides \textsubscript
\ifnum 0\ifxetex 1\fi\ifluatex 1\fi=0 % if pdftex
  \usepackage[T1]{fontenc}
  \usepackage[utf8]{inputenc}
\else % if luatex or xelatex
  \ifxetex
    \usepackage{mathspec}
  \else
    \usepackage{fontspec}
  \fi
  \defaultfontfeatures{Ligatures=TeX,Scale=MatchLowercase}
\fi
% use upquote if available, for straight quotes in verbatim environments
\IfFileExists{upquote.sty}{\usepackage{upquote}}{}
% use microtype if available
\IfFileExists{microtype.sty}{%
\usepackage[]{microtype}
\UseMicrotypeSet[protrusion]{basicmath} % disable protrusion for tt fonts
}{}
\PassOptionsToPackage{hyphens}{url} % url is loaded by hyperref
\usepackage[unicode=true]{hyperref}
\PassOptionsToPackage{usenames,dvipsnames}{color} % color is loaded by hyperref
\hypersetup{
            pdftitle={Webscraping in R},
            colorlinks=true,
            linkcolor=blue,
            citecolor=red,
            urlcolor=blue,
            breaklinks=true}
\urlstyle{same}  % don't use monospace font for urls
\usepackage[margin=1in]{geometry}
\usepackage[]{biblatex}
\usepackage{color}
\usepackage{fancyvrb}
\newcommand{\VerbBar}{|}
\newcommand{\VERB}{\Verb[commandchars=\\\{\}]}
\DefineVerbatimEnvironment{Highlighting}{Verbatim}{commandchars=\\\{\}}
% Add ',fontsize=\small' for more characters per line
\usepackage{framed}
\definecolor{shadecolor}{RGB}{248,248,248}
\newenvironment{Shaded}{\begin{snugshade}}{\end{snugshade}}
\newcommand{\KeywordTok}[1]{\textcolor[rgb]{0.13,0.29,0.53}{\textbf{#1}}}
\newcommand{\DataTypeTok}[1]{\textcolor[rgb]{0.13,0.29,0.53}{#1}}
\newcommand{\DecValTok}[1]{\textcolor[rgb]{0.00,0.00,0.81}{#1}}
\newcommand{\BaseNTok}[1]{\textcolor[rgb]{0.00,0.00,0.81}{#1}}
\newcommand{\FloatTok}[1]{\textcolor[rgb]{0.00,0.00,0.81}{#1}}
\newcommand{\ConstantTok}[1]{\textcolor[rgb]{0.00,0.00,0.00}{#1}}
\newcommand{\CharTok}[1]{\textcolor[rgb]{0.31,0.60,0.02}{#1}}
\newcommand{\SpecialCharTok}[1]{\textcolor[rgb]{0.00,0.00,0.00}{#1}}
\newcommand{\StringTok}[1]{\textcolor[rgb]{0.31,0.60,0.02}{#1}}
\newcommand{\VerbatimStringTok}[1]{\textcolor[rgb]{0.31,0.60,0.02}{#1}}
\newcommand{\SpecialStringTok}[1]{\textcolor[rgb]{0.31,0.60,0.02}{#1}}
\newcommand{\ImportTok}[1]{#1}
\newcommand{\CommentTok}[1]{\textcolor[rgb]{0.56,0.35,0.01}{\textit{#1}}}
\newcommand{\DocumentationTok}[1]{\textcolor[rgb]{0.56,0.35,0.01}{\textbf{\textit{#1}}}}
\newcommand{\AnnotationTok}[1]{\textcolor[rgb]{0.56,0.35,0.01}{\textbf{\textit{#1}}}}
\newcommand{\CommentVarTok}[1]{\textcolor[rgb]{0.56,0.35,0.01}{\textbf{\textit{#1}}}}
\newcommand{\OtherTok}[1]{\textcolor[rgb]{0.56,0.35,0.01}{#1}}
\newcommand{\FunctionTok}[1]{\textcolor[rgb]{0.00,0.00,0.00}{#1}}
\newcommand{\VariableTok}[1]{\textcolor[rgb]{0.00,0.00,0.00}{#1}}
\newcommand{\ControlFlowTok}[1]{\textcolor[rgb]{0.13,0.29,0.53}{\textbf{#1}}}
\newcommand{\OperatorTok}[1]{\textcolor[rgb]{0.81,0.36,0.00}{\textbf{#1}}}
\newcommand{\BuiltInTok}[1]{#1}
\newcommand{\ExtensionTok}[1]{#1}
\newcommand{\PreprocessorTok}[1]{\textcolor[rgb]{0.56,0.35,0.01}{\textit{#1}}}
\newcommand{\AttributeTok}[1]{\textcolor[rgb]{0.77,0.63,0.00}{#1}}
\newcommand{\RegionMarkerTok}[1]{#1}
\newcommand{\InformationTok}[1]{\textcolor[rgb]{0.56,0.35,0.01}{\textbf{\textit{#1}}}}
\newcommand{\WarningTok}[1]{\textcolor[rgb]{0.56,0.35,0.01}{\textbf{\textit{#1}}}}
\newcommand{\AlertTok}[1]{\textcolor[rgb]{0.94,0.16,0.16}{#1}}
\newcommand{\ErrorTok}[1]{\textcolor[rgb]{0.64,0.00,0.00}{\textbf{#1}}}
\newcommand{\NormalTok}[1]{#1}
\IfFileExists{parskip.sty}{%
\usepackage{parskip}
}{% else
\setlength{\parindent}{0pt}
\setlength{\parskip}{6pt plus 2pt minus 1pt}
}
\setlength{\emergencystretch}{3em}  % prevent overfull lines
\providecommand{\tightlist}{%
  \setlength{\itemsep}{0pt}\setlength{\parskip}{0pt}}
\setcounter{secnumdepth}{0}
% Redefines (sub)paragraphs to behave more like sections
\ifx\paragraph\undefined\else
\let\oldparagraph\paragraph
\renewcommand{\paragraph}[1]{\oldparagraph{#1}\mbox{}}
\fi
\ifx\subparagraph\undefined\else
\let\oldsubparagraph\subparagraph
\renewcommand{\subparagraph}[1]{\oldsubparagraph{#1}\mbox{}}
\fi

% set default figure placement to htbp
\makeatletter
\def\fps@figure{htbp}
\makeatother


\title{Webscraping in R}
\author{Matthew
Malishev\textsuperscript{1}*\\[2\baselineskip]\emph{\textsuperscript{1}
Department of Biology, Emory University, 1510 Clifton Road NE, Atlanta,
GA, USA, 30322}}
\date{}

\begin{document}
\maketitle

{
\hypersetup{linkcolor=black}
\setcounter{tocdepth}{4}
\tableofcontents
}
\newpage   

Date: 2018-11-12\\
R version: 3.5.0\\
*Corresponding author:
\href{mailto:matthew.malishev@gmail.com}{\nolinkurl{matthew.malishev@gmail.com}}\\
This document can be found at \url{https://github.com/darwinanddavis}

~

R session info

\begin{Shaded}
\begin{Highlighting}[]
\NormalTok{params}\OperatorTok{$}\NormalTok{session}
\end{Highlighting}
\end{Shaded}

\begin{verbatim}
R version 3.5.0 (2018-04-23)
Platform: x86_64-apple-darwin15.6.0 (64-bit)
Running under: OS X El Capitan 10.11.6

Matrix products: default
BLAS: /Library/Frameworks/R.framework/Versions/3.5/Resources/lib/libRblas.0.dylib
LAPACK: /Library/Frameworks/R.framework/Versions/3.5/Resources/lib/libRlapack.dylib

locale:
[1] en_US.UTF-8/en_US.UTF-8/en_US.UTF-8/C/en_US.UTF-8/en_US.UTF-8

attached base packages:
[1] stats     graphics  grDevices utils     datasets  methods   base     

loaded via a namespace (and not attached):
 [1] compiler_3.5.0  backports_1.1.2 magrittr_1.5    rprojroot_1.3-2 tools_3.5.0     htmltools_0.3.6
 [7] pillar_1.2.3    tibble_1.4.2    yaml_2.2.0      Rcpp_0.12.19    stringi_1.2.3   rmarkdown_1.10 
[13] knitr_1.20      stringr_1.3.1   digest_0.6.15   rlang_0.3.0.1   evaluate_0.10.1
\end{verbatim}

\newpage  

\subsection{Overview}\label{overview}

This document outlines useful tools for webscraping with \texttt{R},
including how to use xpaths, navigating XML and JSON, and useful
\texttt{R} packages.

\subsubsection{Install dependencies}\label{install-dependencies}

\begin{Shaded}
\begin{Highlighting}[]
\NormalTok{packages <-}\StringTok{ }\KeywordTok{c}\NormalTok{(}\StringTok{"rgdal"}\NormalTok{,}\StringTok{"dplyr"}\NormalTok{,}\StringTok{"zoo"}\NormalTok{,}\StringTok{"RColorBrewer"}\NormalTok{,}\StringTok{"viridis"}\NormalTok{,}\StringTok{"plyr"}\NormalTok{,}\StringTok{"digitize"}\NormalTok{,}\StringTok{"jpeg"}\NormalTok{,}\StringTok{"devtools"}\NormalTok{,}\StringTok{"imager"}\NormalTok{,}\StringTok{"dplyr"}\NormalTok{,}\StringTok{"ggplot2"}\NormalTok{,}\StringTok{"ggridges"}\NormalTok{,}\StringTok{"ggjoy"}\NormalTok{,}\StringTok{"ggthemes"}\NormalTok{,}\StringTok{"svDialogs"}\NormalTok{,}\StringTok{"data.table"}\NormalTok{,}\StringTok{"tibble"}\NormalTok{,}\StringTok{"extrafont"}\NormalTok{,}\StringTok{"sp"}\NormalTok{)   }
\ControlFlowTok{if}\NormalTok{ (}\KeywordTok{require}\NormalTok{(packages)) \{}
    \KeywordTok{install.packages}\NormalTok{(packages,}\DataTypeTok{dependencies =}\NormalTok{ T)}
    \KeywordTok{require}\NormalTok{(packages)}
\NormalTok{\}}
\KeywordTok{lapply}\NormalTok{(packages,library,}\DataTypeTok{character.only=}\NormalTok{T)}
\end{Highlighting}
\end{Shaded}

\subsection{Initial steps}\label{initial-steps}

\begin{enumerate}
\def\labelenumi{\arabic{enumi}.}
\tightlist
\item
  Search CRAN for packages that access the API for the site for
  webscraping, e.g.~search `Spotify' for \texttt{Rspotify} package.\\
\item
  Use \texttt{rvest} or \texttt{rjson} to scrape.\\
\item
  Use \texttt{regex} functions to parse text blocks.
\end{enumerate}

\subsection{Selecting individual webpage
elements}\label{selecting-individual-webpage-elements}

xpath finder: webpage plugin to isolate HTML elements, e.g.~tables on
webpage and turn them into XML. Once passed to \texttt{R}, packages like
\texttt{rvest} scrape only that XML element.

Use \texttt{tidytext} in R to parse and clean scraped text.

\subsection{Geocodes}\label{geocodes}

\begin{enumerate}
\def\labelenumi{\arabic{enumi}.}
\tightlist
\item
  Use Google API to get geocode values from webpage.\\
\item
  Run xpath expression to isolate the geocode values e.g ``XML parse''
  function.\\
  \#\# PubMed Search pubmed in \texttt{R} package on CRAN.\\
  Use the HTML element web app to find which HTML elements of the page
  you want to scrape, e.g. = link.\\
  To scrape the pure text of i.e.~just abstract text without the author
  and affiliation text lines, but across multiple web pages, use the XML
  search home page to search for and apply an xpath term, e.g.
  `abstracttextonly'.
\end{enumerate}

\subsubsection{\texorpdfstring{How to programmatically bypass the `Show
more results' button on
webpages}{How to programmatically bypass the Show more results button on webpages}}\label{how-to-programmatically-bypass-the-show-more-results-button-on-webpages}

\begin{enumerate}
\def\labelenumi{\arabic{enumi}.}
\tightlist
\item
  Use \texttt{RSelenium} package.\\
\item
  Find the HTML element on the webpage for the `Show More' button and
  access this XML path using the package.
\end{enumerate}

\printbibliography

\end{document}
